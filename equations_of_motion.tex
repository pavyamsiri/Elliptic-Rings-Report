\section{Equations of Motion}
Currently we only have the torque exerted on a ring. In order to obtain the equations of motion governing the orientation of the
ring, we need the relationship between torque and angular acceleration. Torque is defined as
\begin{equation}
    \vec{\tau} = \dbyd{\vec{L}}{t}\label{eq:torque_angular_momentum}
\end{equation}
the time derivative of the angular momentum \( \vec{L} \) which itself is given by
\begin{equation}
    \vec{L} = \mathbf{I} \vec{\omega}\label{eq:angular_momentum}
\end{equation}
where \( \mathbf{I} \) is the called the inertia tensor and \( \vec{\omega} \) is the angular velocity vector.
Substituting Eqn.~\ref{eq:angular_momentum} into
Eqn.~\ref{eq:torque_angular_momentum} gives
\begin{align}
    \vec{\tau} & = \dbyd{}{t} \left(\mathbf{I} \vec{\omega}\right) \nonumber                                                 \\
               & = \dbyd{\mathbf{I}}{t} \vec{\omega} + \mathbf{I} \dbyd{\vec{\omega}}{t} \nonumber                           \\
               & = \dbyd{\mathbf{I}}{t} \vec{\omega} + \mathbf{I} \vec{\alpha} \label{eq:torque_angular_momentum_expression}
\end{align}
where \( \vec{\alpha} \) is the angular acceleration vector. In order to get an expression for the term \( \dbyd{\mathbf{I}}{t} \),
we can express the inertia tensor as being a rotation of a constant inertia tensor \( \tilde{\mathbf{I}} \) calculated in a fixed
inertial frame of the ring
\begin{equation}
    \mathbf{I} = \mathbf{R} \tilde{\mathbf{I}} {\mathbf{R}}^{T}
\end{equation}
where \( \mathbf{R} \) is the rotation matrix that transform from the fixed inertial frame to our current frame. Taking the time
derivative of this expression gives
\begin{align}
    \dbyd{\mathbf{I}}{t} =
    \dbyd{\mathbf{R}}{t} \tilde{\mathbf{I}} {\mathbf{R}}^{T}
    + \mathbf{R} \tilde{\mathbf{I}} \dbyd{{\mathbf{R}}^{T}}{t} \label{eq:inertia_tensor_time_derivative}
\end{align}
The time derivative of the rotation matrix can be written in terms of the angular velocity \( \vec{\omega} \)
\begin{equation}
    \dbyd{\mathbf{R}}{t} = {\mathbf{\omega}}_{\times} \mathbf{R} \label{eq:rotation_matrix_time_derivative}
\end{equation}
where \( {\mathbf{\omega}}_{\times} \) is a skew-symmetric matrix defined as the map \( \vec{x} \mapsto \vec{\omega} \times \vec{x} \),
i.e.~matrix representation of the cross product with \( \vec{\omega} \). The components are
\begin{equation}
    {\mathbf{\omega}}_{\times} = \begin{pmatrix}
        0         & -\omega_z & \omega_y  \\
        \omega_z  & 0         & -\omega_x \\
        -\omega_y & \omega_x  & 0
    \end{pmatrix}
\end{equation}
Substituting Eqn.~\ref{eq:rotation_matrix_time_derivative} into Eqn.~\ref{eq:inertia_tensor_time_derivative} gives
\begin{align}
    \dbyd{\mathbf{I}}{t}
     & = \dbyd{\mathbf{R}}{t} \tilde{\mathbf{I}} {\mathbf{R}}^{T}
    + \mathbf{R} \tilde{\mathbf{I}} {\left(\dbyd{\mathbf{R}}{t}\right)}^{T} \nonumber                               \\
     & = \left({\mathbf{\omega}}_{\times} \mathbf{R}\right) \tilde{\mathbf{I}} {\mathbf{R}}^{T}
    + \mathbf{R} \tilde{\mathbf{I}} {\left({\mathbf{\omega}}_{\times} \mathbf{R}\right)}^{T} \nonumber              \\
     & = {\mathbf{\omega}}_{\times} \left(\mathbf{R}\tilde{\mathbf{I}}{\mathbf{R}}^{T}\right)
    + \mathbf{R} \tilde{\mathbf{I}} {\left({\mathbf{R}}^{T} {\mathbf{\omega}_{\times}}^{T}\right)} \nonumber        \\
     & = {\mathbf{\omega}}_{\times} \mathbf{I}
    + \left(\mathbf{R} \tilde{\mathbf{I}} {\mathbf{R}}^{T} \right) \left(-\mathbf{\omega}_{\times}\right) \nonumber \\
    \implies \dbyd{\mathbf{I}}{t}
     & = {\mathbf{\omega}}_{\times} \mathbf{I}
    - \mathbf{I}{\mathbf{\omega}}_{\times} \label{eq:inertia_tensor_time_derivative_expression}
\end{align}
Using the Eqn.~\ref{eq:inertia_tensor_time_derivative_expression} in Eqn.~\ref{eq:torque_angular_momentum_expression} gets
\begin{align}
    \vec{\tau}          & = \left({\mathbf{\omega}}_{\times} \mathbf{I} - \mathbf{I}{\mathbf{\omega}}_{\times}\right) \vec{\omega}
    + \mathbf{I} \vec{\alpha} \nonumber                                                                                            \\
                        & = {\mathbf{\omega}}_{\times} \mathbf{I} \vec{\omega} - \mathbf{I}{\mathbf{\omega}}_{\times} \vec{\omega}
    + \mathbf{I} \vec{\alpha} \nonumber                                                                                            \\
                        & = {\mathbf{\omega}}_{\times} \vec{L} - \mathbf{I}\left(\vec{\omega} \times \vec{\omega}\right)
    + \mathbf{I} \vec{\alpha} \nonumber                                                                                            \\
    \implies \vec{\tau} & = \vec{\omega} \times \vec{L} + \mathbf{I} \vec{\alpha}
\end{align}
Now rearrange to get an expression for the angular acceleration
\begin{equation}
    \dbyd{\vec{\omega}}{t} = \vec{\alpha}
    = {\mathbf{I}}^{-1} \left(\vec{\tau} - \vec{\omega} \times \vec{L}\right)
        = {\mathbf{I}}^{-1} \left[\vec{\tau} - \vec{\omega} \times \left(\mathbf{I} \times \vec{\omega}\right)\right]
\end{equation}
Currently, we are missing the differentual equation for the orientation of the ring. This is not such a simple task as
integrating the angular velocity with respect to time due to the intricacies of rotations in 3D space. See
Section~\ref{sec:quaternions} to see how angular velocity will be integrated.

\subsection{Inertia Tensor}
In this section we will derive inertia tensor of an elliptic ring about its centre of mass in the ring's frame of reference.
In this frame of reference, the ring lies on the \( x-y \)-plane and the \( z \)-axis is perpendicular to the plane of the ring.
The \( x \)-axis will be taken to be the major axis of the ellipse and the \( y \)-axis will be taken to be the minor axis of the
ellipse.

The inertia tensor is a rank 2 symmetric tensor which can be written in matrix form as
\begin{equation}
    \mathbf{I} = \begin{pmatrix}
        I_{xx} & I_{xy} & I_{xz} \\
        I_{yx} & I_{yy} & I_{yz} \\
        I_{zx} & I_{zy} & I_{zz}
    \end{pmatrix}
\end{equation}
The diagonals are the moments of inertia about the \( x \), \( y \) and \( z \) axes respectively.
The off-diagonals are the products of inertia. The products of inertia are zero if the object is rotationally symmetric about all
frame axes, and so in this frame of reference, the inertia tensor is diagonal
\begin{equation}
    \mathbf{I} = \begin{pmatrix}
        I_{xx} & 0      & 0      \\
        0      & I_{yy} & 0      \\
        0      & 0      & I_{zz}
    \end{pmatrix}
\end{equation}
The moment of inertia \( I \) about an axis \( P \) is the given by the equation
\begin{equation}
    I_P = \iiint_{Q} \rho(\vec{r}) {\lVert\vec{r}\rVert}^{2} \,{}\,{} \mathrm{d}V
\end{equation}
where \( \vec{r} \) is the distance from the axis \( P \) to a point in \( Q \)
and \( \rho(\vec{r}) \) is the mass density at that point.
We will first derive the density distribution of the ring and then calculate the moments of inertia about the \( x,y,z \)
axes in terms of the \( M, a, b \) and \( R \).
\subsubsection{Density}
Let the elliptic ring be defined by the parametric equations over the parameter \( \theta \)
\begin{align}
    x & = aR \cos{\theta} \\
    y & = bR \sin{\theta} \\
    z & = 0
\end{align}
where \( a \) is the semi-major axis scale length, \( b \) is the semi-minor axis scale length, \( R \) is the scale radius of the
ellipse. In terms of this parametrisation, the cylindrical radial coordinate \( r \) of the ring is given by
\begin{equation}
    r = \frac{bR}{\sqrt{1 - e^2{\cos}^{2}{\theta}}}
\end{equation}
This means the mass density has the form
\begin{equation}
    \rho(\vec{r}) = \rho(r, \theta, z) = \rho_0 \delta\left(r - \frac{bR}{\sqrt{1 - e^2 \cos^2{\theta}}}\right) \delta(z)
\end{equation}
where the coefficient \( \rho_0 \) is a normalisation constant which can be written in terms of the total mass by integrating the
mass density over the entire ring to get the total mass \( M \)
\begin{align}
    M & = \int_{0}^{\infty} \int_{0}^{2\pi} \int_{-\infty}^{\infty} \rho(r, \theta, z) \,{}\,{} r \rd{r} \rd{\theta} \rd{z} \nonumber                                                               \\
      & = \rho_0 \int_{0}^{\infty} \int_{0}^{2\pi} \int_{-\infty}^{\infty} \delta\left(r - \frac{bR}{\sqrt{1 - e^2 \cos^2{\theta}}}\right) \delta(z) \,{}\,{} r \rd{r} \rd{\theta} \rd{z} \nonumber \\
      & = \rho_0 \int_{0}^{\infty} \int_{0}^{2\pi} \delta\left(r - \frac{bR}{\sqrt{1 - e^2 \cos^2{\theta}}}\right) \,{}\,{} r \rd{r} \rd{\theta} \nonumber                                          \\
      & = bR\rho_0 \int_{0}^{2\pi} \frac{1}{\sqrt{1-e^2\cos^2{\theta}}} \,{}\,{} \mathrm{d}\theta \nonumber
\end{align}
The integral can be written in terms of the complete elliptic integral of the first kind \( K \) which is defined as
\begin{equation}
    K(k) = \int_{0}^{\frac{\pi}{2}} \frac{1}{\sqrt{1- k \sin^2{\theta}}} \,{}\,{} \mathrm{d}\theta
\end{equation}
and so
\begin{align}
    M               & = bR\rho_0 \cdot 4 K\left(e^2\right) \nonumber \\
                    & = 4 b R \rho_0 K\left(e^2\right) \nonumber     \\
    \implies \rho_0 & = \frac{M}{4bR K\left(e^2\right)}
\end{align}
Hence, we have
\begin{equation}
    \rho(\vec{r}) = \rho(r, \theta, z) = \frac{M}{4bRK\left(e^2\right)}
    \delta\left(r - \frac{bR}{\sqrt{1 - e^2 \cos^2{\theta}}}\right) \delta(z) \label{density}
\end{equation}

\subsubsection{Moment of inertia about the \texorpdfstring{\( z \)}{z}-axis}
The moment of inertia about the \( z \)-axis is given by
\begin{equation}
    I_{zz} = \int_{0}^{\infty} \int_{0}^{2\pi} \int_{-\infty}^{\infty} \rho(r, \theta, z) \left(x^2 + y^2\right) \cdot \,{}\,{} r \rd{r}\rd{\theta}\rd{z}
\end{equation}
Substituting in Eqn~\ref{density} gets us
\begin{align}
    I_{zz} & = \frac{M}{4bRK\left(e^2\right)} \int_{0}^{\infty} \int_{0}^{2\pi} \delta\left(r - \frac{bR}{\sqrt{1 - e^2{\cos}^{2}{\theta}}}\right) r^3 \,{}\,{} \rd{r}\rd{\theta} \nonumber \\
           & = \frac{M}{4bRK\left(e^2\right)} \int_{0}^{2\pi} \frac{b^3 R^3}{{\left(1 - e^2 \cos^2{\theta}\right)}^{\frac{3}{2}}} \,{}\,{} \rd{\theta} \nonumber                            \\
           & = \frac{M b^2 R^2}{4K\left(e^2\right)} \cdot 4 \int_{0}^{\frac{\pi}{2}} \frac{1}{{\left(1 - e^2 \cos^2{\theta}\right)}^{\frac{3}{2}}} \,{}\,{} \rd{\theta} \nonumber           \\
           & = M \frac{b^2 R^2}{K\left(e^2\right)} \int_{0}^{\frac{\pi}{2}} \frac{1}{{\left(1 - e^2 \cos^2{\theta}\right)}^{\frac{3}{2}}} \,{}\,{} \rd{\theta} \nonumber
\end{align}
The integral can be written in terms of the complete elliptic integral of the second kind
which is defined as
\begin{equation}
    E(k) = \int_{0}^{\frac{\pi}{2}} \sqrt{1 - k \sin^2{\theta}} \,{}\,{} \mathrm{d}\theta
\end{equation}
and so
\begin{align}
    I_{zz}          & = M \frac{b^2 R^2}{K\left(e^2\right)} \cdot \frac{E\left(e^2\right)}{1 - e^2}  \nonumber                              \\
                    & = M \frac{b^2 R^2}{K\left(e^2\right)} \cdot \frac{E\left(e^2\right)}{1 - \left(1 - \frac{b^2}{a^2}\right)}  \nonumber \\
                    & = M \frac{b^2 R^2}{K\left(e^2\right)} \cdot \frac{E\left(e^2\right)}{\frac{b^2}{a^2}}  \nonumber                      \\
    \implies I_{zz} & = M \frac{E\left(e^2\right)}{K\left(e^2\right)} a^2 R^2
\end{align}
% TODO Plot E(e^2)/K(e^2)

\subsubsection{Moment of inertia about the \texorpdfstring{\( x \)}{x}-axis}
The moment of inertia about the \( x \)-axis is
\begin{equation}
    I_{xx} = \int_{0}^{\infty} \int_{0}^{2\pi} \int_{-\infty}^{\infty} \rho(r, \theta, z) \left(y^2 + z^2\right) \cdot \,{}\,{} r \rd{r}\rd{\theta}\rd{z}
\end{equation}
Performing the same procedure we did for \( I_{zz} \) gives
\begin{align}
    I_{xx} & = \frac{M}{4bRK\left(e^2\right)} \int_{0}^{\infty} \int_{0}^{2\pi} \delta\left(r - \frac{bR}{\sqrt{1 - e^2\cos^2{\theta}}}\right) y^2 r \,{}\,{} \rd{r}\rd{\theta} \nonumber             \\
           & = \frac{M}{4bRK\left(e^2\right)} \int_{0}^{\infty} \int_{0}^{2\pi} \delta\left(r - \frac{bR}{\sqrt{1 - e^2\cos^2{\theta}}}\right) r^3\sin^2{\theta} \,{}\,{} \rd{r}\rd{\theta} \nonumber \\
           & = \frac{Mb^2R^2}{4K\left(e^2\right)}  \int_{0}^{2\pi} \frac{\sin^2{\theta}}{{\left({1 - e^2\cos^2{\theta}}\right)}^{\frac{3}{2}}}  \,{}\,{} \rd{\theta} \nonumber                        \\
           & = \frac{Mb^2R^2}{K\left(e^2\right)}  \int_{0}^{\frac{\pi}{2}} \frac{\sin^2{\theta}}{{\left({1 - e^2\cos^2{\theta}}\right)}^{\frac{3}{2}}}  \,{}\,{} \rd{\theta}
\end{align}
The integral again can be written in terms of elliptic integrals
\begin{equation}
    \int_{0}^{\frac{\pi}{2}} \frac{\sin^2{\theta}}{{\left({1 - e^2\cos^2{\theta}}\right)}^{\frac{3}{2}}}  \,{}\,{} \rd{\theta}
    = \frac{K\left(-{e'}^{2}\right) -\left(1 - e^2\right)E\left(-{e'}^{2}\right)}{e^2\sqrt{1 - e^2}}
\end{equation}
where \( e' \) is called the second eccentricity and is defined in terms of the eccentricity as
\begin{equation}
    e' = \frac{e}{\sqrt{1 - e^2}}
\end{equation}
therefore
\begin{equation}
    I_{xx} = M
    \left[
        \frac{K\left(-{e'}^{2}\right) -\left(1 - e^2\right)E\left(-{e'}^{2}\right)}{e^2K\left(e^2\right)}
        \right]
    ab R^2
\end{equation}
where we used the fact that \( \sqrt{1 - e^2} = \frac{b}{a} \).

\subsubsection{Moment of inertia about the \texorpdfstring{\( y \)}{y}-axis}
The moment of inertia about the \( y \)-axis is
\begin{equation}
    I_{yy} = \int_{0}^{\infty} \int_{0}^{2\pi} \int_{-\infty}^{\infty} \rho(r, \theta, z) \left(x^2 + z^2\right) \cdot \,{}\,{} r \rd{r}\rd{\theta}\rd{z}
\end{equation}
Following the pattern from the previous two moments of inertia, we get that
\begin{equation}
    I_{yy} = \frac{Mb^2R^2}{K\left(e^2\right)}  \int_{0}^{\frac{\pi}{2}} \frac{\cos^2{\theta}}{{\left({1 - e^2\cos^2{\theta}}\right)}^{\frac{3}{2}}}  \,{}\,{} \rd{\theta}
\end{equation}
Again using elliptic integrals, the integral can be expressed as
\begin{equation}
    \int_{0}^{\frac{\pi}{2}} \frac{\cos^2{\theta}}{{\left({1 - e^2\cos^2{\theta}}\right)}^{\frac{3}{2}}}  \,{}\,{} \rd{\theta}
    = \frac{E\left(-{e'}^{2}\right) - K\left(-{e'}^{2}\right)}{e^2\sqrt{1 - e^2}}
\end{equation}
and so
\begin{equation}
    I_{yy} = M
    \left[
        \frac{E\left(-{e'}^{2}\right) - K\left(-{e'}^{2}\right)}{e^2K\left(e^2\right)}
        \right]
    ab R^2
\end{equation}

\subsection{Integrating angular velocity \texorpdfstring{\( \omega \)}{w}}\label{sec:quaternions}
Rotations in 3D are not commutative meaning the order of rotations matters. This is in contrast to 2D rotations which are
commutative. A standard way to represent rotations in 3D is to use Euler angles. Euler angles are a set of three angles which
define a series of three rotations about a set of fixed axes. A common set of Euler angles are the yaw, pitch and roll angles used
in aviation. Euler angles suffer from a problem called gimbal lock. This occurs when two rotation axes become aligned, leading
to a reduction in the dimensionality of the space of possible rotations.

\subsubsection{Quaternions}
An alternative to Euler angles are quaternions which are a four dimensional extension of the complex numbers. Quaternions are
not subject to gimbal lock and are more computationally efficient than Euler angles. A quaternion \( q \) has the form
\begin{equation}
    \mathbf{q} = a + b\mathbf{i} + c\mathbf{j} + d\mathbf{k}
\end{equation}
where \( a, b, c \) and \( d \) are real numbers and \( \mathbf{i}, \mathbf{j} \) and \( \mathbf{k} \) are the basis elements.
The basis elements satisfy the following multiplication rules
\begin{align}
    \mathbf{i}^2 = \mathbf{j}^2 = \mathbf{k}^2 = \mathbf{ijk} = -1
\end{align}
A quaternion \( \mathbf{q} = a + b\mathbf{i} + c\mathbf{j} + d\mathbf{k}  \) can be decomposed into two parts, a scalar
component \( a \) and a vector component \( \vec{v} = \begin{pmatrix} b \\ c \\ d \end{pmatrix} \). A pure or vector quaternion
is a quarternion with no scalar component.

Let two quaternions have the components
\begin{align}
    \mathbf{q}_1 & = \begin{pmatrix}
                         w_1 \\ \vec{v}_1
                     \end{pmatrix} \\
    \mathbf{q}_2 & = \begin{pmatrix}
                         w_2 \\ \vec{v}_2
                     \end{pmatrix}
\end{align}
The sum of two quarternions is simply an element-wise addition of the components
\begin{equation}
    \mathbf{q}_1 + \mathbf{q}_2 = \begin{pmatrix}
        w_1 + w_2 \\ \vec{v}_1 + \vec{v}_2
    \end{pmatrix}
\end{equation}
The product of two quaternions is non-commutative and can be expressed as
\begin{equation}
    \mathbf{q}_1 \times \mathbf{q}_2 = \begin{pmatrix}
        w_1 w_2 - \vec{v}_1 \cdot \vec{v}_2 \\
        w_1 \vec{v}_2 + w_2 \vec{v}_1 + \vec{v}_1 \times \vec{v}_2
    \end{pmatrix}
\end{equation}
The norm of a quaternion is defined as
\begin{equation}
    \lVert \mathbf{q} \rVert = \sqrt{a^2 + b^2 + c^2 + d^2}
\end{equation}
The conjugate \( \mathbf{q}^{*} \) of a quaternion \( \mathbf{q} \) is a quaternion that satisfies the following relation
\begin{equation*}
    \mathbf{q} \times \mathbf{q}^{*} = {\lVert \mathbf{q} \rVert}^2
\end{equation*}
and can be expressed as
\begin{equation}
    \mathbf{q}^* = \begin{pmatrix}
        w \\ -\vec{v}
    \end{pmatrix}
\end{equation}
The inverse \( \mathbf{q}^{-1} \) of a quaternion \( \mathbf{q} \) is the quaternion that when multiplied gives the identity
quarternion
\begin{equation}
    \mathbf{q} \times \mathbf{q}^{-1} = (1, 0, 0, 0)
\end{equation}
Note that from now on the identity quarternion will be simply written as \( 1 \).
The inverse can be written as the conjugate divided by the square of the norm
\begin{equation}
    \mathbf{q}^{-1} = \frac{\mathbf{q}^{*}}{{\lVert \mathbf{q} \rVert}^2}
\end{equation}
therefore a unit quaternion's inverse is simply its conjugate.

A rotation in 3D can be represented by a quaternion \( \mathbf{q} \) of unit length. If the axis of rotation is given by the unit
vector
\begin{equation}
    \mathbf{u} = \begin{pmatrix}
        u_x \\ u_y \\ u_z
    \end{pmatrix}
\end{equation}
and the angle of rotation is \( \theta \), the corresponding quaternion that represents this rotation has components
\begin{align}
    q = \begin{pmatrix}
            \cos{\frac{\theta}{2}}     \\
            u_x \sin{\frac{\theta}{2}} \\
            u_y \sin{\frac{\theta}{2}} \\
            u_z \sin{\frac{\theta}{2}}
        \end{pmatrix}
\end{align}
In order to rotate a vector \( \vec{v} \) about this axis by the angle \( \theta \), we simply use the formula
\begin{equation}
    \mathbf{v}' = \mathbf{q} \times \mathbf{v} \times \mathbf{q}^{-1} \label{eq:quaternion_rotation}
\end{equation}
where \( \mathbf{v} \) is simply a vector quarternion with its vector component equal to \( \vec{v} \). The rotated vector
\( \vec{v}' \) is the vector component of the quaternion \( \mathbf{v}' \).

Taking the time derivative of Eqn.~\ref{eq:quaternion_rotation} gives
\begin{equation}
    \dbyd{\mathbf{v}'}{t} = \dbyd{\mathbf{q}}{t} \times \mathbf{v} \times \mathbf{q}^{-1}
    + \mathbf{q} \times \mathbf{v} \times \dbyd{\mathbf{q}^{-1}}{t}\label{eq:quaternion_rotation_derivative}
\end{equation}
The term \( \dbyd{\mathbf{q}^{-1}}{t} \) can be found by differentiating the equation \( 1 = \mathbf{q} \times \mathbf{q}^{-1} \)
\begin{align}
    0
     & = \dbyd{\mathbf{q}}{t} \times \mathbf{q}^{-1} + \mathbf{q} \times \dbyd{\mathbf{q}^{-1}}{t} \nonumber          \\
    \implies \dbyd{\mathbf{q}^{-1}}{t}
     & = -\mathbf{q}^{-1} \times \dbyd{\mathbf{q}}{t} \times \mathbf{q}^{-1} \label{eq:quaternion_inverse_derivative}
\end{align}
Substituting Eqn.~\ref{eq:quaternion_inverse_derivative} into Eqn.~\ref{eq:quaternion_rotation_derivative} gives
\begin{align}
    \dbyd{\mathbf{v}'}{t}
                                   & = \dbyd{\mathbf{q}}{t} \times \mathbf{v} \times \mathbf{q}^{-1}
    - \mathbf{q} \times \mathbf{v} \times \mathbf{q}^{-1} \times \dbyd{\mathbf{q}}{t} \times \mathbf{q}^{-1} \nonumber                                      \\
                                   & = \dbyd{\mathbf{q}}{t} \times \left(\mathbf{q}^{-1} \times \mathbf{v}' \times \mathbf{q}\right) \times \mathbf{q}^{-1}
    - \mathbf{v}' \times \dbyd{\mathbf{q}}{t} \times \mathbf{q}^{-1} \nonumber                                                                              \\
                                   & = \dbyd{\mathbf{q}}{t} \times \mathbf{q}^{-1} \times \mathbf{v}'
    - \mathbf{v}' \times \dbyd{\mathbf{q}}{t} \times \mathbf{q}^{-1} \nonumber                                                                              \\
    \implies \dbyd{\mathbf{v}'}{t} & = \left[\dbyd{\mathbf{q}}{t}\times\mathbf{q}^{-1}, \mathbf{v}'\right]\label{eq:quaternion_time_derivative_commutator}
\end{align}
where the square brackets denote the usual commutator. It can be shown that \( \dbyd{\mathbf{q}}{t}\times\mathbf{q}^{-1} \) is a
vector quarternion if and only if \( \mathbf{q} \times \mathbf{q}^{*} \) is constant over time. This means
Eqn.~\ref{eq:quaternion_time_derivative_commutator} can be expressed as
\begin{equation}
    \dbyd{\mathbf{v}'}{t} = \left(2\dbyd{\mathbf{q}}{t}\times \mathbf{q}^{-1}\right) \times \mathbf{v}'
\end{equation}
Now as both \( \dbyd{\mathbf{q}}{t}\times \mathbf{q}^{-1} \) and \( \mathbf{v}' \) are vector quarternions the last product is
equivalent to usual vector cross product.
Recall that the time derivative of a rotating vector \( \vec{v}' \) is equal to the cross product of its angular velocity vector
and itself
\begin{equation}
    \dbyd{\vec{v}'}{t} = \vec{\omega} \times \vec{v}'
\end{equation}
Hence we have
\begin{equation}
    \mathbf{\omega} = 2\dbyd{\mathbf{q}}{t}\times \mathbf{q}^{-1}
\end{equation}
where we have turned \( \vec{\omega} \) into a vector quarternion.
Rearrange to get a differential equation for \( \mathbf{q} \)
\begin{equation}
    \dbyd{\mathbf{q}}{t} = \frac{1}{2} \mathbf{\omega} \times \mathbf{q}
\end{equation}
We have now obtained all the equations of motion for the system
\begin{align}
    \dbyd{\mathbf{q}}{t}   & = \frac{1}{2} \mathbf{\omega} \times \mathbf{q}\label{eq:quaternion_eom}                                      \\
    \dbyd{\vec{\omega}}{t} & = {\mathbf{I}}^{-1} \left[\vec{\tau} - \vec{\omega} \times \left(\mathbf{I} \times \vec{\omega}\right)\right]
\end{align}
where \( \mathbf{q} \) is the quaternion representing the orientation of the ring and \( \vec{\omega} \) is the angular velocity.

\subsection{Numerical Scheme}
We will now derive a numerical scheme to solve the equations of motion. We will first integrate angular velocity via a first
order scheme
\begin{equation}
    \vec{\omega}_{n+1} = \vec{\omega}_n + \vec{\alpha}_n \Delta t
\end{equation}
where \( \vec{\alpha}_n \) is the angular acceleration at time \( t_n \) and \( \Delta t \) is the time step.
The angular acceleration is computed by
\begin{equation}
    \vec{\alpha}_n = {\mathbf{I}}^{-1} \left[\vec{\tau}_n - \vec{\omega}_n \times \left(\mathbf{I} \times \vec{\omega}_n\right)\right]
\end{equation}
The inertia tensors are dependent on the current orientation of the ring
\begin{equation}
    \mathbf{I}(\mathbf{q}) = \mathbf{R}(\mathbf{q}) \mathbf{I}_0 {\mathbf{R}(\mathbf{q})}^T
\end{equation}
with the rotation matrices being derived from the quarternion by the equation
\begin{equation}
    \mathbf{R} = \begin{pmatrix}
        1 + 2v_{1} + 2w^2      & 2(v_{1}v_{2} - v_{3}w) & 2(v_{1}v_{3} + v_{2}w) \\
        2(v_{1}v_{2} + v_{3}w) & -1 + 2v_{2}^2 + 2w^2   & 2(v_{2}v_{3} - v_{1}w) \\
        2(v_{1}v_{3} - v_{2}w) & 2(v_{2}v_{3} + v_{1}w) & -1 + 2v_{3}^2 + 2w^2
    \end{pmatrix}
\end{equation}
The differential equation for the quarternion (Eqn.~\ref{eq:quaternion_eom}) in the case of zero angular acceleration \( \alpha = 0 \)
has the solution
\begin{equation}
    \mathbf{q}_{n+1} = \exp\left(\frac{1}{2}\omega\Delta t\right) \times \mathbf{q}_n
\end{equation}
However when \( \alpha \neq 0 \), the solution is more complicated. The solution can be expressed using the Magnus expansion
\begin{equation}
    \mathbf{q}_{n+1} = \exp\left(\frac{1}{2}\Omega\right) \mathbf{q}_{n}
\end{equation}
where \( \mathbf{\Omega} \) is a vector quarternion given by an infinite series
\begin{equation}
    \mathbf{\Omega} = \sum_{k=1}^{\infty} \mathbf{\Omega}_k
\end{equation}
The first few terms of which are
\begin{align}
    \Omega_{1} & = \frac{1}{2}\left(\vec{\omega}_{n} + \vec{\omega}_{n+1}\right)\Delta t                                                   \\
    \Omega_{2} & = \frac{1}{12}\left(\vec{\omega}_{n+1} \times \vec{\omega}_{n}\right){\Delta t}^{2}                                       \\
    \Omega_{3} & = \frac{1}{240}\left[\vec{\alpha}_{n} \times \left(\vec{\alpha}_{n} \times \vec{\omega}_{n+1}\right)\right]{\Delta t}^{5}
\end{align}
The quarternion exponential is defined by the infinite series
\begin{equation}
    \exp\left(\mathbf{q}\right) = \sum_{n=0}^{\infty} \frac{\mathbf{q}^{n}}{n!}
\end{equation}
but in the special case of a vector quarternion
\begin{equation*}
    \exp\left(\mathbf{q}\right) = \cos\left(\lVert\mathbf{q}\rVert\right) + \frac{\sin\left(\lVert\mathbf{q}\rVert\right)}{\lVert\mathbf{q}\rVert}\mathbf{q}
\end{equation*}
% TODO: These Magnus terms are computed for the case where alpha = constant.